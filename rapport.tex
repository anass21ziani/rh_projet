\documentclass[12pt,a4paper]{article}
\usepackage[utf8]{inputenc}
\usepackage[french]{babel}
\usepackage{geometry}
\usepackage{graphicx}
\usepackage{booktabs}
\usepackage{array}
\usepackage{float}
\usepackage{hyperref}
\usepackage{xcolor}
\usepackage{listings}
\usepackage{enumitem}
\usepackage{titlesec}

\geometry{margin=2.5cm}
\hypersetup{
    colorlinks=true,
    linkcolor=blue,
    filecolor=magenta,      
    urlcolor=cyan,
}

\title{\textbf{Rapport de Développement - Système de Gestion RH}}
\author{Développement d'un Système de Gestion des Ressources Humaines}
\date{\today}

\begin{document}

\maketitle

\tableofcontents
\newpage

\section{Introduction}

Ce rapport présente l'évolution et les améliorations apportées au système de gestion des ressources humaines développé avec Symfony 7.3. Le système a été enrichi de nouvelles fonctionnalités, d'une interface utilisateur modernisée et d'un système de tableaux de bord adaptatifs selon les rôles des utilisateurs.

\section{Architecture du Système}

\subsection{Technologies Utilisées}
Le système repose sur une architecture moderne utilisant :
\begin{itemize}
    \item \textbf{Symfony 7.3} : Framework PHP principal
    \item \textbf{PostgreSQL 16} : Base de données relationnelle
    \item \textbf{Doctrine ORM 3.5} : Mapping objet-relationnel
    \item \textbf{Bootstrap 5} : Framework CSS pour l'interface utilisateur
    \item \textbf{Twig} : Moteur de templates
    \item \textbf{Chart.js} : Bibliothèque JavaScript pour les graphiques
\end{itemize}

\subsection{Architecture MVC}
Le système suit le pattern Model-View-Controller avec :
\begin{itemize}
    \item \textbf{Modèles} : Entités Doctrine représentant les données métier
    \item \textbf{Vues} : Templates Twig pour l'affichage
    \item \textbf{Contrôleurs} : Logique métier et gestion des requêtes
\end{itemize}

\section{Nouvelles Fonctionnalités Implémentées}

\subsection{Système de Gestion des Placards}

\subsubsection{Objectif}
Intégration d'un système de gestion des placards physiques pour l'organisation des dossiers employés dans l'espace de travail.

\subsubsection{Fonctionnalités}
\begin{itemize}
    \item Création et gestion des placards avec identification unique
    \item Association des dossiers employés aux placards spécifiques
    \item Interface de gestion dédiée pour les responsables RH
    \item Visualisation de l'occupation des placards
\end{itemize}

\subsubsection{Impact}
Cette fonctionnalité améliore l'organisation physique des documents et facilite la localisation des dossiers employés dans l'environnement de travail.

\subsection{Système de Demandes et Réclamations}

\subsubsection{Objectif}
Mise en place d'un canal de communication structuré entre les employés et les responsables RH pour les demandes et réclamations.

\subsubsection{Fonctionnalités}
\begin{itemize}
    \item Création de demandes par les employés
    \item Système de suivi des demandes avec statuts
    \item Interface de gestion pour les responsables RH
    \item Notifications et réponses aux demandes
\end{itemize}

\subsubsection{Impact}
Améliore la communication interne et permet un suivi structuré des demandes des employés.

\subsection{Tableaux de Bord Adaptatifs}

\subsubsection{Objectif}
Création de tableaux de bord personnalisés selon le rôle de l'utilisateur connecté.

\subsubsection{Fonctionnalités par Rôle}

\paragraph{Dashboard Responsable RH}
\begin{itemize}
    \item Indicateurs de performance globaux (KPIs)
    \item Graphiques de répartition des employés par département
    \item Statistiques des contrats (actifs, expirés, suspendus)
    \item Vue d'ensemble des dossiers et documents
    \item Graphiques interactifs avec Chart.js
\end{itemize}

\paragraph{Dashboard Employé}
\begin{itemize}
    \item KPIs personnels (contrats, dossiers, documents, demandes)
    \item Graphiques des statuts de contrats personnels
    \item Graphiques des statuts de dossiers personnels
    \item Informations personnelles centralisées
    \item Actions rapides vers les fonctionnalités principales
\end{itemize}

\subsubsection{Impact}
Améliore l'expérience utilisateur en fournissant des informations pertinentes selon le contexte et le rôle de chaque utilisateur.

\section{Améliorations de l'Interface Utilisateur}

\subsection{Modernisation du Design}

\subsubsection{Éléments Visuels}
\begin{itemize}
    \item Interface moderne avec Bootstrap 5
    \item Cartes KPI avec bordures colorées et icônes
    \item Graphiques interactifs et responsifs
    \item Palette de couleurs cohérente
    \item Typographie améliorée
\end{itemize}

\subsubsection{Navigation}
\begin{itemize}
    \item Barre de navigation latérale avec modules organisés
    \item Navigation adaptative selon les rôles
    \item Liens d'actions rapides
    \item Breadcrumbs pour la navigation contextuelle
\end{itemize}

\subsection{Responsivité}
L'interface s'adapte automatiquement aux différentes tailles d'écran, garantissant une expérience optimale sur desktop, tablette et mobile.

\section{Gestion des Statuts et Workflows}

\subsection{Statuts des Dossiers}
\begin{itemize}
    \item \textbf{En attente} : Dossier créé, en attente de traitement
    \item \textbf{En cours} : Dossier en cours de traitement
    \item \textbf{Complété} : Dossier finalisé
\end{itemize}

\subsection{Statuts des Contrats}
\begin{itemize}
    \item \textbf{Actif} : Contrat en cours de validité
    \item \textbf{Expiré} : Contrat arrivé à expiration
    \item \textbf{Suspendu} : Contrat temporairement suspendu
\end{itemize}

\section{Sécurité et Contrôle d'Accès}

\subsection{Système de Rôles}
Le système implémente un contrôle d'accès basé sur les rôles :
\begin{itemize}
    \item \textbf{ROLE\_ADMINISTRATEUR\_RH} : Accès complet au système
    \item \textbf{ROLE\_RESPONSABLE\_RH} : Gestion des employés et dossiers
    \item \textbf{ROLE\_EMPLOYEE} : Accès aux informations personnelles
\end{itemize}

\subsection{Sécurité}
\begin{itemize}
    \item Authentification sécurisée
    \item Protection CSRF sur les formulaires
    \item Headers de sécurité HTTP
    \item Validation des données côté serveur
\end{itemize}

\section{Base de Données et Entités}

\subsection{Nouvelles Tables}
\begin{itemize}
    \item \textbf{placard} : Gestion des placards physiques
    \item \textbf{demande} : Système de demandes et réclamations
    \item \textbf{type\_document} : Types de documents standardisés
    \item \textbf{type\_document\_nature\_contrat} : Association types de documents et natures de contrats
\end{itemize}

\subsection{Relations}
Les entités sont liées par des relations Doctrine appropriées :
\begin{itemize}
    \item Relations OneToMany et ManyToOne
    \item Relations ManyToMany pour les associations complexes
    \item Contraintes d'intégrité référentielle
\end{itemize}

\section{Performance et Optimisation}

\subsection{Optimisations Implémentées}
\begin{itemize}
    \item Mise en cache des requêtes fréquentes
    \item Lazy loading pour les relations Doctrine
    \item Compression des assets
    \item Optimisation des requêtes de base de données
\end{itemize}

\subsection{Métriques de Performance}
\begin{itemize}
    \item Temps de chargement des pages optimisé
    \item Réduction de la charge serveur
    \item Amélioration de l'expérience utilisateur
\end{itemize}

\section{Expérience Utilisateur}

\subsection{Workflow Employé}
\begin{enumerate}
    \item Connexion au système
    \item Accès au dashboard personnel
    \item Consultation des informations personnelles
    \item Gestion des demandes et réclamations
    \item Accès aux documents et contrats
\end{enumerate}

\subsection{Workflow Responsable RH}
\begin{enumerate}
    \item Connexion au système
    \item Accès au dashboard global
    \item Gestion des employés et dossiers
    \item Traitement des demandes
    \item Gestion des placards
    \item Consultation des statistiques
\end{enumerate}

\section{Intégration et Compatibilité}

\subsection{Compatibilité Navigateur}
Le système est compatible avec les navigateurs modernes :
\begin{itemize}
    \item Chrome 90+
    \item Firefox 88+
    \item Safari 14+
    \item Edge 90+
\end{itemize}

\subsection{Responsive Design}
L'interface s'adapte aux différentes résolutions :
\begin{itemize}
    \item Desktop (1920x1080 et plus)
    \item Laptop (1366x768)
    \item Tablet (768x1024)
    \item Mobile (375x667)
\end{itemize}

\section{Maintenance et Évolutivité}

\subsection{Architecture Modulaire}
Le système est conçu pour faciliter la maintenance :
\begin{itemize}
    \item Séparation claire des responsabilités
    \item Code modulaire et réutilisable
    \item Documentation intégrée
    \item Tests unitaires et fonctionnels
\end{itemize}

\subsection{Évolutivité}
L'architecture permet l'ajout facile de nouvelles fonctionnalités :
\begin{itemize}
    \item Système de plugins
    \item API REST pour intégrations futures
    \item Base de données extensible
    \item Interface utilisateur modulaire
\end{itemize}

\section{Conclusion}

\subsection{Bilan des Améliorations}
Les développements réalisés ont considérablement enrichi le système de gestion RH :

\begin{itemize}
    \item \textbf{Fonctionnalités} : Ajout de 4 nouvelles entités et de leurs interfaces de gestion
    \item \textbf{Interface} : Modernisation complète avec des tableaux de bord adaptatifs
    \item \textbf{Expérience utilisateur} : Amélioration significative avec des KPIs visuels et des graphiques interactifs
    \item \textbf{Organisation} : Système de placards pour une meilleure gestion physique des documents
    \item \textbf{Communication} : Canal structuré pour les demandes et réclamations
\end{itemize}

\subsection{Perspectives d'Évolution}
Le système est maintenant prêt pour de futures évolutions :
\begin{itemize}
    \item Intégration d'un système de notifications en temps réel
    \item Développement d'une application mobile
    \item Intégration avec des systèmes externes (paie, formation)
    \item Mise en place d'un système de workflow avancé
    \item Ajout de fonctionnalités d'analyse prédictive
\end{itemize}

\subsection{Impact Métier}
Ces améliorations apportent une valeur ajoutée significative :
\begin{itemize}
    \item Amélioration de l'efficacité opérationnelle
    \item Réduction des temps de traitement des demandes
    \item Meilleure organisation des documents physiques
    \item Amélioration de la communication interne
    \item Tableaux de bord pour le pilotage et la décision
\end{itemize}

Le système de gestion RH est maintenant un outil complet et moderne, adapté aux besoins actuels et futurs de l'organisation.

\end{document}
